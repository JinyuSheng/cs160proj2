% !TEX encoding =  UTF-8 Unicode
% !TEX TS-program = lilypond-book
% !LILYPOND tex = xelatex
\documentclass{article}
\usepackage{ifxetex}
\ifxetex
%xetex specific stuff
\usepackage{xunicode,fontspec,xltxtra}
\setmainfont[Numbers=OldStyle]{Times New Roman}
\setsansfont{Arial}
\else
%This can be empty if you are not going to use pdftex
\usepackage[T1]{fontenc}
\usepackage[utf8]{inputenc}
\usepackage{mathptmx}%Times
\usepackage{helvet}%Helvetica
\fi
%Here you can insert all packages that pdftex also understands
\usepackage[ngerman,finnish,english]{babel}
\usepackage{graphicx}

\begin{document}
\title{A short document with LilyPond and xelatex}
\maketitle

Normal \textbf{font} commands inside the \emph{text} work,
because they \textsf{are supported by \LaTeX{} and XeteX.}
If you want to use specific commands like \verb+\XeTeX+, you
should include them again in a \verb+\ifxetex+ environment.
You can use this to print the \ifxetex \XeTeX{} command \else
XeTeX command \fi which is not known to normal \LaTeX .

In normal text you can easily use LilyPond commands, like this:

\begin{lilypond}
{a2 b c'8 c' c' c'}
\end{lilypond}

\noindent
and so on.

The fonts of snippets set with LilyPond will have to be set from
inside
of the snippet. For this you should read the AU on how to use
lilypond-book.

\selectlanguage{ngerman}
Auch Umlaute funktionieren ohne die \LaTeX -Befehle, wie auch alle
anderen
seltsamen Zeichen: \ldots, wenn sie von der Schriftart
unterst..tzt werden.
\end{document}
\end
