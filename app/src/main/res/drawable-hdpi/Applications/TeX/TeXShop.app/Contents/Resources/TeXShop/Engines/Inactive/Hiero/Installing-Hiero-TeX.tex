1) Download HieroTeXMac.tar.bz2 and HieroTypeMac.tar.bz2 from the web site 

	http://www.filipvervloesem.be/hierotexosx/Site/How_to_install_HieroTeX_on_Mac_OS_X.html

2) Create a folder named Hiero on your desktop. Put both files in this folder. Double click both files. This will create two subfolders,

	HieroTeXMac
	texmf

3) TeX contains a vast number of files: style files, class files, fonts, etc. These files are organized in a tree of directories called a "texmf tree"; many directories in this tree have standard names which cannot be changed. 

For instance, LaTeX looks for files in

     texmf/tex/latex

or any subfolder of this folder. Files can be placed directly in the latex folder, but it is considered polite to put files in a subfolder with an appropriate name. Hiero is going to put its files for LaTeX in

     texmf/tex/latex/hierotex

Similarly fonts for TeX go into subfolders of

     texmf/fonts
     
The version of texmf supplied by the Hiero web site didn't handle font files in quite the way TeX Live expects. It put all the information for the Hiero fonts in 

     texmf/fonts/type1

but TeX Live wants the pfa files to be in texmf/fonts/type1, the tfm files to be in texmf/fonts/tfm/hierotex, the afm files to be in texmf/fonts/afm/hierotex, and the map files to be in texmf/fonts/map/hierotex.

In addition to files and folders for the texmf tree, the Hiero distribution contains one binary file named sesh. Writing a HieroTeX document involves editing a document with extension .htx, running this document through sesh to create a .tex document, and typesetting this document with LaTeX.

The following instructions will automatically rearrange the HieroTeXMac and texmf folders in the folder Hiero created in step 2. After the commands run, you will have one texmf tree and one additional sesh file. To run these instructions, open Terminal and change directory to the Hiero folder.

	cd 
	cd Desktop/Hiero
	
Then copy the entire chain of commands below and paste them into Terminal. (Retyping the commands rather than pasting them will inevitably lead to errors.)

	cd texmf
	mkdir -p fonts/afm/hierotex
	mkdir -p fonts/tfm/hierotex
	mkdir -p fonts/map/hiero

	cd dvips
	mv *.map ../fonts/map/hiero
	cd ..

	cd fonts/type1
	mv *.afm ../afm/hierotex
	mv *.tfm ../tfm/hierotex

	mkdir hierotex
	mv *.pfa hierotex
	mv Fontmap hierotex
	mv psfonts.map hierotex
	mv HieroE.pfm hierotex

	cd ..
	cd ..
	cd ..
	
	cd HieroTeXMac
	
	cd texmf
	cp -Rv doc ../../texmf
	cd fonts
	cp -Rv source ../../../texmf/fonts
	cd ..
	cp -Rv tex ../../texmf

	cd ..
	mv sesh ..
	
	cd ..
	rm -R HieroTeXMac
	
4) At this point, it is necessary to merge the hierotex texmf tree we just created with the rest of the TeX Live texmf tree. TeX Live has several texmf trees. Some are in the official distribution, and aren't supposed to be touched. Another tree is provided for local additions which every user can access, and yet another tree is for local additions for a particular user.

The tree for all users on the local machine is at /usr/local/texlive/texmf-local. Initially this tree is empty, but you may have already added files there. To merge the Hiero tree with the existing tree, change directory in Terminal to the texmf folder inside the Hiero folder. 

	cd
	cd Desktop/Hiero/texmf

Then issue the following command. You will be asked for an administrative password.

	sudo cp -Rv *  /usr/local/texlive/texmf-local/

5) Aside: you could also put the Hiero tree in your personal texmf tree in ~/Library/texmf. Here ~/Library is the Library folder in your home directory, and you may have to create a texmf folder inside if one isn't already there. Since you own this folder, you can just drag files from the Hiero texmf tree to this tree. Or you can modify the previous command and write

	cd
	cd Desktop/Hiero/texmf
	cp -Rv *  ~/Library/texmf
	
I recommend the method in 4) instead.

6) Next we will put sesh in the TeX Live binary directory. The command below assumes that you installed MacTeX. In that case you have a TeX Distribution preference pane which shows up in Apple's System Preferences. Make sure that the appropriate distribution is selected in this pane (presumably TeX Live 2008 or TeX Live 2009) and then issue the Terminal commands
	
	cd 
	cd Desktop/Hiero
	sudo cp -Rv sesh /usr/texbin
	
7) To speed up searching for files, TeX Live creates an index of its various texmf trees. Since we added files to these trees, we must revise the index. This isn't necessary when we add files to ~/Library/texmf. But otherwise, go to Terminal and type

	sudo texhash
	
8) We now need to tell TeX about the new fonts we installed. Go to Terminal and type the following commands:

	sudo updmap-sys --enable Map=diacrFonts

	sudo updmap-sys --enable Map=hierofonts
	
At this point, Hiero TeX should be working.
	
Incidentally, the Hiero web site suggests directly editing the file /usr/local/texlive/2007/texmf-var/web2c/updmap.cfg. This isn't necessary or desirable, and in fact this file doesn't exist in TeX Live 2009.
	
10) To tell TeXShop about HieroTeX, go to ~/Library/TeXShop/Engines/Inactive/Hiero and drag Hiero.engine two levels up to ~/Library/TeXShop/Engines.



	
	
	